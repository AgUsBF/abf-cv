\documentclass[a4paper]{../data/myCV}

% improve word spacing and hyphenation
\usepackage{microtype}
\usepackage{ragged2e}

% enable mathematical syntax for some symbols like \varnothing
\usepackage{amssymb}

% uncomment in case you don't want any hyphenation
% \usepackage[none]{hyphenat}

% take care of proper font encoding
\ifxetexorluatex
	\usepackage{fontspec}
	\defaultfontfeatures{Ligatures=TeX}
\else
	\usepackage[utf8]{inputenc}
	\usepackage[T1]{fontenc}
\fi

% language
\usepackage[spanish]{babel}


%-------------------------------------------------------------------------------
%                            PERSONAL INFORMATION
%-------------------------------------------------------------------------------

\cvname{Agust\'in\\Beceyro Ferr\'an}

\cvjobtitle{Mag\'ister en Ingenier\'ia\\Ingeniero Mec\'anico}

\cvprofilepic{../data/pics/ID.jpg}

\cvaddress{
	CABA,
	Argentina
	\flag{../data/pics/flags/AR.png}
}

\cvmail{
	agustinbeceyro@gmail.com
}

\cvcustomdata{\faLinkedinIn}{
	\href{https://www.linkedin.com/in/agustin-bf}{
		Agust\'in Beceyro Ferr\'an
	}
}

\cvcustomdata{\faGithub}{
	\href{https://github.com/AgUsBF}{
		GitHub
	}
}


%-------------------------------------------------------------------------------
%                              SIDEBAR 1st PAGE
%-------------------------------------------------------------------------------

\addtofrontsidebar{

	\profilesection{Perfil}
	\aboutme{
		Soy un profesional con experiencia en ingenier\'ia e investigaci\'on y 
		desarrollo.
		Actualmente, estoy ampliando mi perfil incorporando competencias en 
		desarrollo de software, an\'alisis de datos y metodolog\'ias \'agiles.\\

		Colabor\'e en diversos proyectos de ingeniería, incluyendo sistemas de 
		propulsi\'on espacial, un cohete sonda, un difract\'ometro de neutrones y un 
		electrolizador de alta presi\'on.
		Adem\'as, mi recorrido en el \'ambito acad\'emico me permiti\'o desarrollar 
		habilidades tanto en investigaci\'on como en docencia.\\

		El manejo fluido del ingl\'es me ha permitido colaborar con equipos 
		multidisciplinarios internacionales.
		Como parte de mi compromiso con el crecimiento continuo, he complementado mi 
		formaci\'on con estudios de posgrado y diversas capacitaciones.\\

		Busco un nuevo desaf\'io profesional donde pueda aportar mi experiencia y 
		seguir evolucionando, contribuyendo al logro de los objetivos de la 
		organizaci\'on.
	}
}


%-------------------------------------------------------------------------------
%                         TABLE ENTRIES RIGHT COLUMN
%-------------------------------------------------------------------------------
\begin{document}

\makefrontsidebar

\cvsection{Experiencia Laboral}
\begin{cvtable}[3]
	\cvitem{2021 -- 2025}{Ingeniero de I+D + Project Manager}{LIA Aerospace}{
		Coordinaci\'on del desarrollo de un sistema de propulsi\'on espacial.
		Otras tareas: an\'alisis preliminar de mec\'anica orbital, pruebas de motores
		cohete, an\'alisis de datos, redacci\'on y revisi\'on de informes t\'ecnicos,
		planificaci\'on de seguridad	y operaciones en el sitio de pruebas de motores.
	}  

	\cvitem{2019 -- 2021}{Ingeniero de I+D}{LIA Aerospace}{
		Dise\~no, integraci\'on y prueba de sistemas de vuelo y terrestres, prueba de
		motores cohete, an\'alisis de datos, simulaci\'ones de trayectoria de un
		cohete sonda, redacci\'on y revisi\'on de informes t\'ecnicos.
	}

	\cvitem{2015 -- 2019}{Ingeniero de I+D}{Centro At\'omico Bariloche - CNEA}{
		Simulaci\'on de la \'optica de un difract\'ometro de neutrones, desarrollo de
		c\'odigo en Python para el an\'alisis de datos y rutinas de optimizaci\'on
		num\'erica, redacci\'on y revisi\'on de informes t\'ecnicos.
	}

	\cvitem{2014 -- 2015}{Becario de Posgrado}{Centro At\'omico Constituyentes - CNEA}{
		Beneficiario de una beca para el posgrado en Reactores Nucleares y su Ciclo 
		de Combustible del Instituto Dan Beninson.
	}

	\cvitem{2012 -- 2014}{Ingeniero de I+D}{Instituto Tecnol\'ogico de Buenos Aires}{
		Colaboraci\'on en dos proyectos: instalaci\'on y prueba de un electrolizador 
		de alta presi\'on en un complejo industrial, y el dise\~no, construcci\'on y 
		prueba de un sistema de propulsi\'on para un sat\'elite.\\
		Participaci\'on en actividades acad\'emicas como ayudante de trabajos pr\'acticos, 
		co-tutor de tesis de gradoy mentor de pasantes.
	}
	
	\cvitem{2008 -- 2012}{Pasante de I+D}{Instituto Tecnol\'ogico de Buenos Aires}{
		Dise\~no, construcci\'on y prueba de prototipos en el Laboratorio de
		Hidr\'ogeno del Departamento de Ingenier\'ia Mec\'anica.
	}
	
	\cvitem{2007 -- 2008}{T\'ecnico en Impresi\'on Offset}{Gr\'afica ECO}{
		Operaci\'on, calibraci\'on y mantenimiento de impresoras offset.
	}
\end{cvtable}

\vfill

\cvsection{Educaci\'on}
\begin{cvtable}[1.5]
	\cvitem{2024 -- }{Analista de Sistemas}{Escuela Da Vinci}{
		Promedio parcial: 9.9/10.
	}
		
	\cvitem{2017 -- 2019}{Maestr\'ia en Ingenier\'ia}{Instituto Balseiro}{
		Foco: \'Optica de Neutrones.
		Promedio: 9.6/10.\\
		Tesis: Dise\~no conceptual y estimaci\'on de performance del difract\'ometro
		de neutrones ANDES para el estudio de muestras en polvo.
	}

	\cvitem{2014}{Especializaci\'on}{Instituto Dan Beninson}{
		Reactores Nucleares y su Ciclo de Combustible.
		Promedio: 9.6/10.\\
		Tesis: An\'alisis de diferentes formas de c\'alculo de dosis en terapias 
		por captura neutr\'onica utilizando MCNP.
	}

	\cvitem{2005 -- 2012}{Ingenier\'ia Mec\'anica}{Instituto Tecnol\'ogico de Buenos Aires}{
		Premio a mejor tesis de Inger\'ia Mec\'anica 2012.
		Promedio: 7.3/10.\\
		Tesis: Dise\~no y construci\'on  de un electrolizador de alta presi\'on.\\
	}
		
	\cvitem{2002 -- 2004}{Bachiller Biling\"ue}{Balmoral College}{
		Orientacion: Ciencias Naturales.
		Promedio: 8.0/10.
	}
\end{cvtable}

\vfill


%-------------------------------------------------------------------------------
%                              SIDEBAR 2nd PAGE
%-------------------------------------------------------------------------------
\addtobacksidebar{

	\profilesection{Herramientas}
	\skill{\faCode}{
	Python,
	Java,
	PHP,
	HTML,
	CSS,
	SQL
}
		
\skill{\faTools}{
	Git,
	Jupyter Notebooks,
	Jira,
	Trello
}
		
\skill{\faLinux}{
	Linux,
	macOS,
	Windows
}
		
\skill{\faAlignJustify}{
	Latex,
	Markdown
}

%\barskill{\faCode}{Python}{50}
	
	% Flags from https://github.com/gosquared/flags;
	\graphicspath{{../data/pics/flags/}}
	\profilesection{Idiomas}
	\skill{\flag{AR.png}}{Espa\~nol}
	\skill{\flag{GB.png}}{Ingl\'es}
	\skill[1.8em]{\faGraduationCap}{
	Certificate in Advanced English
}

\skill[1.8em]{\faGraduationCap}{
	IGCSE,
	GCE (AS Level)
}

\skill[1.8em]{\faGraduationCap}{
	First Certificate in English
}

\skill[1.8em]{\faGraduationCap}{
	TOEIC
}
}


%-------------------------------------------------------------------------------
%                                  2nd PAGE
%-------------------------------------------------------------------------------
\newpage

\makebacksidebar

\cvsection{Cursos}
\begin{cvtable}
	\cvitem{2020}{IBM Data Science}{IBM + edx}{}

	\cvitem{2020}{Python Data Science}{IBM + edx}{}

	\cvitem{2020}{Data Science Foundations}{IBM + edx}{}

	\cvitem{2018}{Introduction to Computing in Python}{Georgia Tech + edx}{}

	\cvitem{2017}{Introducci\'on al lenguaje Python orientado\\
					a ingenier\'ias y f\'isica}{Instituto Balseiro}{}
\end{cvtable}



\cvsection{Publicaciones}
\begin{cvtable}
	\cvpubitem{Estudio de factibilidad de un sistema de propulsi\'on de gas fr\'io para cubesat 12U}
	{A. Beceyro Ferr\'an, A. M. Caratozzolo}
	{XIII Congreso Argentino de Tecnolog\'ia Espacial}
	{2025}

	\cvpubitem{Optimizaci\'on de la \'optica neutr\'onica de ANDES en los modos de alta intensidad y resoluci\'on media}
	{A. Beceyro Ferr\'an, J.R. Santisteban, M.A. Vicente \'Alvarez}
	{Segundo Congreso Argentino de T\'ecnicas Neutr\'onicas}
	{2019}

	\cvpubitem{Aplicaciones del difract\'ometro ANDES para el LAHN}
	{M.A. Vicente \'Alvarez, A. Beceyro Ferr\'an, A. Moya Riffo, G. Juarez}
	{Segundo Congreso Argentino de T\'ecnicas Neutr\'onicas}
	{2019}

	\cvpubitem{ANDES neutron optics in the powder diffraction mode}
	{A. Beceyro Ferr\'an, J.R. Santisteban, M.A. Vicente \'Alvarez}
	{Primer Congreso Argentino de T\'ecnicas Neutr\'onicas}
	{2017}

	\cvpubitem{ANDES – a multi-purpose neutron diffractometer for LAHN}
	{ M.A. Vicente \'Alvarez, J.R. Santisteban, A. Beceyro Ferr\'an, S. Gomez, J. I. M\'arquez Dami\'an, A. Coleff, A. Glucksberg, L. Montero, S. Pincin}
	{Primer Congreso Argentino de T\'ecnicas Neutr\'onicas}
	{2017}

	\cvpubitem{Preliminary design of a multi-purpose diffractometer for the RA-10 reactor}
	{A. Beceyro Ferr\'an, J.R. Santisteban}
	{Central European Training School - Budapest}
	{2016}

	\cvpubitem{An\'alisis de la \'optica neutr\'onica de un difract\'ometro multiprop\'osito para el reactor RA-10}
	{J.R. Santisteban, A. Beceyro Ferr\'an}
	{Reuni\'on 42 de la Asociaci\'on Argentina de Tecnolog\'ia Nuclear}
	{2015}
\end{cvtable}

\end{document}